\ssr{Заключение}

В ходе выполнения данной курсовой работы было разработано программное обеспечение, моделирующее детскую игрушку <<Лизун>>. Написанное приложение предоставляет пользователю следующие возможности:

\begin{itemize}
	\item растягивать и вдавливать слайм;
	\item изменять такие параметры объекта, как цвет, коэффициент пропускания, показатель поглощения, степень разбиения поверхности, массу, коэффициенты упругости и затухания;
	\item перемещать и поворачивать камеру;
	\item перезагружать сцену.
\end{itemize}

Были решены следующие задачи:
\begin{itemize}
	\item проведены анализ и выбор существующих моделей и алгоритмов, необходимых для реализации программы моделирования детской игрушки <<Лизун>>;
	\item разработано соответствующее программное обеспечение;
	\item проведено исследование по сравнению временной эффективности работы программы при параллельной и последовательной реализациях алгоритма удаления невидимых линий и поверхностей, по результатам которого параллельная реализация работает в три раза быстрее последовательной.
\end{itemize}
