\ssr{Введение}

В настоящее время компьютерная графика имеет широкое применение в
различных сферах. В частности, визуализация объектов с помощью электронно-вычислительных машин используется в киноиндустрии, разработке компьютерных
игр и моделировании физических процессов.

Особое внимание уделяется визуализации деформируемых тел, которые
могут менять свою форму, внутреннюю структуру, объем и площадь поверхности
под действием внешних сил. Примером такой визуализации является моделирование детской игрушки <<Лизун>>.

Целью данной курсовой работы является разработка программного обеспечения, моделирующего детскую игрушку <<Лизун>> как вязкоупругое тело. Сцена должна содержать в себе камеру, пол, имеющий структуру, и <<лизуна>>. Приложение должно предоставлять пользователю возможность изменять объект (растягивать, вдавливать, изменять цвет, степень прозрачности, коэффициент упругости, массу, степень разбиения поверхности и коэффициент затухания) и управлять камерой.

Задачи работы:
\begin{itemize}
	\item провести анализ и выбор существующих моделей и алгоритмов, необходимых для реализации программы моделирования детской игрушки <<Лизун>>;
	\item разработать соответствующее программное обеспечение;
	\item провести исследование по сравнению временной эффективности работы программы при последовательной и параллельной реализациях алгоритма удаления невидимых линий и поверхностей.
\end{itemize}

\clearpage
