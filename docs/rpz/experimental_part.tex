\chapter{Исследовательская часть}

Целью исследования является сравнение временной эффективности последовательной и параллельной реализаций обратной трассировки лучей.

Исследование было проведено на электронной вычислительной машине со следующими характеристиками:

\begin{itemize}
	\item ОС Fedora Linux 36 (Xfce) x86\_64;
	\item ЦП Intel i7-10510U (8) @ 4.900 ГГц;
	\item ОЗУ 8 ГБ.
\end{itemize}

В рамках данного исследования было измерено время работы последовательной и параллельной реализаций алгоритма обратной трассировки лучей. Замеры проводились для одинаковых исходных параметров:
\begin{itemize}
	\item размер сцены --- 734x564;
	
	\item количество граней слайма было равно 20, 80, 320, 1240 и 4960;
	
	\item коэффициент пропускания и показатель поглощения принимали значения 0.3 и 0.001 соответственно;
	
	\item максимальная глубина рекурсии, реализующее алгоритм обратной трассировки лучей, была равна 3;
	
	\item в течение проведения измерений времени работы не было захватов вершин слайма.
\end{itemize}

Для измерения времени работы реализации алгоритма был использован заголовочный файл chrono.

В таблице \ref{restable} приведены результаты измерений.

\begin{table}[H]
	\begin{center}
		\caption{\label{restable}Результаты исследования}
		\begin{tabular}{|c|c|c|}
			\hline
			\bf{Количество граней} & \bf{Среднее время работы} & \bf{Среднее время работы}\\
			\bf{объекта,} & \bf{параллельной} & \bf{последовательной}\\
			\bf{шт.} & \bf{реализации, мс} & \bf{реализации, мс}\\
			\hline
			20 & 233 & 536\\
			\hline
			80 & 739 & 2099\\
			\hline
			320 & 2788 & 7283\\
			\hline
			1240 & 10738 & 31847\\
			\hline
			4960 & 42253 & 135596\\
			\hline
		\end{tabular}
	\end{center}
\end{table}

На рисунке \ref{resgraph} приведен график результатов исследования.

\begin{figure}[H]
	\begin{center}
		\begin{tikzpicture}
			\begin{axis}[
				title style={align=center},
				title = {Сравнение последовательной и параллельной реализаций\\алгоритма обратной трассировки лучей},
				xlabel = {Количество граней, шт.},
				ylabel = {Время работы, мс},
				legend pos = north west,
				legend style={font=\tiny},
				width = \linewidth
				]
				\legend{ 
					Параллельная реализация,
					Последовательная реализация
				};
				\addplot[mark=none, color=blue] coordinates
				{
					(20, 233)
					(80, 739)
					(320, 2788)
					(1240, 10738)
					(4960, 42253)
				};
				\addplot[mark=none, color=red] coordinates
				{
					(20, 536)
					(80, 2099)
					(320, 7283)
					(1240, 31847)
					(4960, 135596)
				};
			\end{axis}
		\end{tikzpicture}
		\caption{\label{resgraph}График результатов исследования}
	\end{center}
\end{figure}

Из полученных результатов видно, что параллельная реализация алгоритма обратной трассировки работает приблизительно в три раза быстрее, чем последовательная. При этом время работы реализаций прямо пропорционально количеству граней моделируемого объекта.

\clearpage
