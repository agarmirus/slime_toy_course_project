\chapter{Исследовательская часть}

\section{Цель исследования}

Целью исследования является сравнение временной эффективности последовательной и параллельной реализаций обратной трассировки лучей.

\section{Технические характеристики электронно-вычислительной машины}

Ниже приведены технические характеристики электронной вычислительной машины, на которой было произведено исследование работы программы:

\begin{itemize}
	\item Fedora Linux 36 (Xfce) x86\_64;
	\item ЦП Intel i7-10510U (8) @ 4.900 ГГц;
	\item ОЗУ 8 ГБ.
\end{itemize}

\section{Описание исследования}

В рамках данного исследования было измерено время работы последовательной и параллельной реализаций алгоритма обратной трассировки лучей.

Основная информация об исследовании:
\begin{itemize}
	\item размер сцены --- 600x370;
	
	\item количество граней слайма было равно 20, 80, 320, 1240, 4960 и 19840;
	
	\item коэффициент пропускания и показатель поглощения принимали значения 0.3 и 0.001 соответственно;
	
	\item максимальная глубина рекурсии, реализующее алгоритм обратной трассировки лучей, была равна 3;
	
	\item во время измерений со стороны пользователя не оказывалось каких-либо воздействий на слайм.
\end{itemize}

Для измерения времени работы реализации алгоритма был использован заголовочный файл chrono.

\section{Результаты исследования}

В таблице \ref{restable} приведены результаты измерений.

\begin{table}[H]
	\begin{center}
		\caption{\label{restable}Результаты исследования}
		\begin{tabular}{|c|c|c|}
			\hline
			\bf{Количество граней} & \bf{Среднее время работы} & \bf{Среднее время работы}\\
			\bf{объекта,} & \bf{последовательной} & \bf{параллельной}\\
			\bf{шт.} & \bf{реализации, мс} & \bf{реализации, мс}\\
			\hline
			20 & 79 & 224\\
			\hline
			80 & 273 & 798\\
			\hline
			320 & 1158 & 3344\\
			\hline
			1240 & 4191 & 13517\\
			\hline
			4960 & 21875 & 56201\\
			\hline
			19840 & 102175 & 288398\\
			\hline
		\end{tabular}
	\end{center}
\end{table}

На рисунке \ref{resgraph} приведен график результатов исследования.

\begin{figure}[H]
	\begin{center}
		\begin{tikzpicture}
			\begin{axis}[
				title style={align=center},
				title = {Сравнение последовательной и параллельной реализаций\\алгоритма обратной трассировки лучей},
				xlabel = {Количество граней, шт.},
				ylabel = {Время работы, мс},
				legend pos = north west,
				legend style={font=\tiny}
				]
				\legend{ 
					Параллельная реализация,
					Последовательная реализация
				};
				\addplot[mark=none, color=blue] coordinates
				{
					(20, 79)
					(80, 273)
					(320, 1158)
					(1240, 4191)
					(4960, 21875)
					(19840, 102175)
				};
				\addplot[mark=none, color=red] coordinates
				{
					(20, 224)
					(80, 798)
					(320, 3344)
					(1240, 13517)
					(4960, 56201)
					(19840, 288398)
				};
			\end{axis}
		\end{tikzpicture}
		\caption{\label{resgraph}График результатов исследования}
	\end{center}
\end{figure}



\section{Вывод}

В данном разделе было измерено время работы параллельной реализации алгоритма обратной трассировки лучей при разных количествах граней слайма. Из полученных результатов видно, что параллельная реализация алгоритма обратной трассировки работает приблизительно в три раза быстрее, чем последовательная. При этом время работы реализаций прямо пропорционально количеству граней моделируемого объекта.

\clearpage
