\chapter{Исследовательская часть}

\section{Цель исследования}

Целью исследования является оценка временной эффективности параллельной реализации обратной трассировки лучей в зависимости от количества граней моделируемого объекта.

\section{Технические характеристики электронно-вычислительной машины}

Ниже приведены технические характеристики электронной вычислительной машины, на которой было произведено исследование работы программы:

\begin{itemize}
	\item Fedora Linux 36 (Xfce) x86\_64;
	\item ЦП Intel i7-10510U (8) @ 4.900 ГГц;
	\item ОЗУ 8 ГБ.
\end{itemize}

\section{Описание исследования}

В рамках данного исследования было изучено, как количество граней слайма влияет на временную эффективность работы параллельной реализации обратной трассировки лучей. 

Количество граней слайма было равно 20, 80, 320, 1240, 4960 и 19840.

Коэффициент пропускания и показатель поглощения принимали значения: 0.3 и 0.001 соответственно.

Максимальная глубина рекурсии, реализующее алгоритм обратной трассировки лучей, равен 3.

Во время измерений со стороны пользователя не оказывалось каких-либо воздействий на слайм.

Для измерения времени работы реализации алгоритма был использован заголовочный файл chrono.

\section{Результаты исследования}

В таблице \ref{restable} приведены результаты измерений.

\begin{table}[H]
	\begin{center}
		\caption{\label{restable}Результаты исследования}
		\begin{tabular}{|c|c|}
			\hline
			\bf{Количество граней объекта, шт.} & \bf{Среднее время работы, мс}\\
			\hline
			20 & 79\\
			\hline
			80 & 273\\
			\hline
			320 & 1158\\
			\hline
			1240 & 4191\\
			\hline
			4960 & 21875\\
			\hline
			19840 & 102175\\
			\hline
		\end{tabular}
	\end{center}
\end{table}

На рисунке \ref{resgraph} приведен график результатов исследования.

\begin{figure}[H]
	\begin{center}
		\begin{tikzpicture}
			\begin{axis}[
				title style={align=center},
				title = {Зависимость времени работы\\параллельной реализации алгоритма\\обратной трассировки лучей\\от количества граней моделируемого объекта},
				xlabel = {Количество граней, шт.},
				ylabel = {Время работы, мс},
				legend pos = north west,
				legend style={font=\tiny}
				]
				\addplot[mark=none, color=blue] coordinates
				{
					(20, 79)
					(80, 273)
					(320, 1158)
					(1240, 4191)
					(4960, 21875)
					(19840, 102175)
				};
			\end{axis}
		\end{tikzpicture}
		\caption{\label{resgraph}График результатов исследования}
	\end{center}
\end{figure}

Из полученных результатов видно, что время работы параллельной реализации алгоритма обратной трассировки лучей линейно возрастает с количеством граней моделируемого объекта.

\section{Вывод}

В данном разделе было измерено время работы параллельной реализации алгоритма обратной трассировки лучей при разных количествах граней слайма. По полученным результатам можно сделать вывод, что время выполнения реализации алгоритма прямо пропорциональна количеству граней моделируемого объекта.

\clearpage
