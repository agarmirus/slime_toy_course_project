\ssr{Введение}

В настоящее время компьютерная графика имеет широкое применение в
различных сферах. В частности, визуализация объектов с помощью электронно-вычислительных машин используется в киноиндустрии, разработке компьютерных
игр и моделировании физических процессов.

Особое внимание уделяется визуализации деформируемых тел, которые
могут менять свою форму, внутреннюю структуру, объем и площадь поверхности
под действием внешних сил. Одним из таких объектов является детская игрушка
<<Лизун>>.

Целью данной курсовой работы является разработка программного обеспечения, моделирующего детскую игрушку <<Лизун>>.

Задачи работы:

\begin{itemize}
	\item выбрать модель тела и алгоритмы, необходимые для реализации программы моделирования детской игрушки <<Лизун>>;
	\item разработать соответствующее программное обеспечение;
	\item измерить среднее время работы реализации алгоритма удаления невидимых линий и поверхностей;
	\item закрепить знания и навыки, приобретенные в ходе изучения курса компьютерной графики.
\end{itemize}

\clearpage
